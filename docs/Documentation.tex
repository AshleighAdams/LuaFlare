\documentclass[a4paper,11pt]{report}
\usepackage[T1]{fontenc}
\usepackage[utf8]{inputenc}
\usepackage{lmodern}
\usepackage{listings}
\usepackage{color}
\usepackage[margin=1.25in]{geometry}

\renewcommand{\familydefault}{\sfdefault}

\definecolor{dkgreen}{rgb}{0,0.6,0}
\definecolor{gray}{rgb}{0.5,0.5,0.5}
\definecolor{red}{rgb}{0.6,0.2,0.2}

\lstdefinelanguage{Lua}%
  {morekeywords={function,for,local,if,then,else,end,elseif,or,and,not,while,do,until,repeat,true,false,nil},%
   sensitive,%
   morecomment=[l]{--},%
   morecomment=[n]{--[[}{]]},%
   morestring=[d]',%
   morestring=[d]"%
  }[keywords,comments,strings]%

\lstset{
  language=Lua,
  tabsize=4,
  columns=flexible,
  basicstyle={\small\ttfamily},
  keywordstyle=\color{blue},
  commentstyle=\color{dkgreen},
  stringstyle=\color{red},
  showstringspaces=false
}

\newcommand{\setlua}{\lstset{language=Lua}}
\newcommand{\sethtml}{\lstset{language=HTML}}
\newcommand{\setnginx}{\lstset{language=sh}}
\newcommand{\inlinecode}{\texttt}

\lstnewenvironment{lua}[0]
  {\setlua}
  {}
\lstnewenvironment{html}[0]
  {\sethtml}
  {}
\lstnewenvironment{nginx}[0]
  {\setnginx}
  {}

\title{LuaServer}
\author{Kate Adams}

\begin{document}

\maketitle
\tableofcontents

\chapter{Command Line Arguments}

\section{--port=<number>}
\subsubsection{Default Value}{8080}
\subsubsection{Allowed Values}{0-65535}
\subsubsection{Description}{Set the port to bind to.}

\section{--threads=<number>}
\subsubsection{Default Value}{2}
\subsubsection{Allowed Values}{0-*}
\subsubsection{Description}{How many threads to create.}

\section{--threads-model=<string>}
\subsubsection{Default Value}{coroutine}
\subsubsection{Allowed Values}{coroutine, fork, pyrate}
\subsubsection{Description}{How will Lua create the threads?}

\section{--host=<string>}
\subsubsection{Default Value}{*}
\subsubsection{Allowed Values}{*}
\subsubsection{Description}{Bind to this address.}

\section{-l, --local}
\subsubsection{Default Value}{false}
\subsubsection{Allowed Values}{true, false}
\subsubsection{Description}{Set the host to "localhost"}

\section{-t, --unit-test}
\subsubsection{Default Value}{false}
\subsubsection{Allowed Values}{true, false}
\subsubsection{Description}{Perform unit tests and quit.}

\section{-h, --help}
\subsubsection{Default Value}{false}
\subsubsection{Allowed Values}{true, false}
\subsubsection{Description}{Show the help information then quit.}

\section{-v, --version}
\subsubsection{Default Value}{false}
\subsubsection{Allowed Values}{true, false}
\subsubsection{Description}{Show the version information then quit.}

\section{--no-reload}
\subsubsection{Default Value}{false}
\subsubsection{Allowed Values}{true, false}
\subsubsection{Description}{Do not automatically reload scripts.}

\section{--max-etag-size=<size>}
\subsubsection{Default Value}{64MiB}
\subsubsection{Allowed Values}{0-* (k, M, G, T, P, E, Z, Y)(i)(B, b)}
\subsubsection{Description}{Maximium size to generate ETags for.}

\chapter{Handle a Page}

\begin{lua}
reqs.AddPattern(host, url_pattern, callback --[[request, response, ...]])
reqs.AddPattern("*", "/hello_world", hello_world)
reqs.AddPattern("host.com", "/hello_host", hello_host)
reqs.AddPattern("*", "user/%d+/message", send_message)

--AddPattern also appends any captures to the function's arguments:
reqs.AddPattern("*", "user/%d+/message", function(req, res, id)
	print("sending message to ", id)
end)
\end{lua}

\chapter{WebSockets}

\begin{lua}
local callbacks, socket = {}, nil
function callbacks.onconnect(client)
	print("client connected")
end
function callbacks.onmessage(client, msg)
	print("client message: " .. msg)
end
function callbacks.ondisconnect(client)
	print("client disconnected")
end
socket = websocket.register("/path", "protocol", callbacks)
\end{lua}

\section{Create a task}

Most of the time, you can't just use the callbacks, you need some sort of loop (such as a game tick) and send data off to the client.  For this, we will use the task scheduler as follows.

\begin{lua}
local function task()
	while true do
		-- do something
		socket:send("It is now " .. os.date())
		coroutine.yield()
	end
end
\end{lua}

\chapter{Templating System}

LuaServer comes with it's own templating system, you can still use \inlinecode{reqest:append(string)} should you choose to (eg, implimenting your own templating system).
The default templating system offers the \inlinecode{tags} namespace.
The general jist is \inlinecode{tag [, attributes][, children]} where attributes is a key-value table, and children is an indexed (array) or empty table (\inlinecode{table.Count(att) != \#att}).

\section{Escaping}
Escaping with the templating system is all done automatically, however, should you need to write HTML stored in text to the templating system, then you should use the tag \inlinecode{tags.NOESCAPE} which will prevent the very next value from being escaped. Here is an example of it in use:

\begin{lua}
tags.html
{
	tags.NOESCAPE, "<script>alert('hi');</script>"
}
\end{lua}

\section{Examples}

\subsection{Example 1 - Simple}

\begin{lua}
tags.p { class = "test" }
{
  "Here, have some ", tags.b{ "boldness" }, "."
}.print()
\end{lua}

\begin{html}
<p class="test">
	Here, have some 
	<b>
		boldness
	</b>
	.
</p>
\end{html}

\subsection{Example 2 - Basic Page}

\begin{lua}
tags.html
{
	tags.head
	{
		tags.title
		{
			"Hello, world"
		}
	},
	tags.body
	{
		tags.div {class = "test"}
		{
			"This is a really nice generation thingy",
			tags.br, tags.br,
			"Do you like my logo?",
			tags.br,
			tags.img {src = "/logo.png"}
		}
	}
}.print()
\end{lua}

\begin{html}
<html>
	<head>
		<title>
			Hello, world
		</title>
	</head>
	<body>
		<div class="test">
			This is a really nice generation thingy
			<br />
			<br />
			Do you like my logo?
			<br />
			<img src="/logo.png" />
		</div>
	</body>
</html>
\end{html}

\subsection{Example 3 - Sections}

\begin{lua}
local template = tags.div {class = "comments"}
{
	tags.span {"Comments:"},
	tags.SECTION
}

template.print(0)
for i = 1, 5 do
	tags.div {class = "comment"}
	{
		tags.span {class = "author"} { "Anon" .. i },
		tags.span {class = "message"} { "This is a test message." }
	}.print()
end
template.print(1)
\end{lua}

\begin{html}
<div class="comments">
	<span>
		Comments:
	</span>
	<div class="comment">
		<span class="author">
			Anon 1
		</span>
		<span class="message">
			This is a test message.
		</span>
	</div>
	<!-- ... -->
	<div class="comment">
		<span class="author">
			Anon 5
		</span>
		<span class="message">
			This is a test message.
		</span>
	</div>
</div>
\end{html}

\subsection{Example 3 - Unpack}

\begin{lua}
local comments = {}
for i = 1, 5 do
	local comment = tags.div {class = "comment"}
	{
		tags.span {class = "author"} { "Anon" .. i },
		tags.span {class = "message"} { "This is a test message." }
	}
	table.insert(comments, comment)
end

local template = tags.div {class = "comments"}
{
	tags.span {"Comments:"},
	unpack(comments)
}
template.print()
\end{lua}

\begin{html}
<div class="comments">
	<span>
		Comments:
	</span>
	<div class="comment">
		<span class="author">
			Anon 1
		</span>
		<span class="message">
			This is a test message.
		</span>
	</div>
	<!--- ... --->
	<div class="comment">
		<span class="author">
			Anon 5
		</span>
		<span class="message">
			This is a test message.
		</span>
	</div>
</div>
\end{html}

\setlua
\section{Overriding Default Handler}

The following code will remove the hook used by reqs, so you can impliment your own if you desire

\chapter{Behind Nginx}

It is recommended that you run LuaServer behind Nginx to prevent many types of attacks, and other things provided by Nginx, such as compression.

\section{Configurations}
These are some configurations that should be placed in \inlinecode{/etc/nginx/sites-available/}, and system linked to \inlinecode{/etc/nginx/sites-enabled/}

\subsection{HTTP}

\begin{lstlisting}
server {
	listen 80;
	listen [::]:80 ipv6only=on;

	server_name localhost;

	location / {
		proxy_set_header X-Forwarded-For $proxy_add_x_forwarded_for;
		proxy_set_header Host $http_host;
		proxy_pass http://localhost:8080;
	}
}
\end{lstlisting}

\subsection{HTTPS}

\setnginx
\begin{lstlisting}
server {
	listen 443 ssl spdy; # remove spdy if your Nginx installation does not support it
	listen [::]:443 ssl spdy ipv6only=on;

	ssl_certificate cert.pem;
	ssl_certificate_key cert.key;

	ssl_session_timeout 5m;

	ssl_protocols SSLv3 TLSv1;
	ssl_ciphers ALL:!ADH:!EXPORT56:RC4+RSA:+HIGH:+MEDIUM:+LOW:+SSLv3:+EXP;
	ssl_prefer_server_ciphers on;

	server_name localhost;

	location / {
		proxy_set_header X-Forwarded-For $proxy_add_x_forwarded_for;
		proxy_set_header Host $http_host;
		proxy_set_header X-Forwarded-Ssl on;
		proxy_pass http://localhost:8080;
	}
}
\end{lstlisting}

\chapter{Internal Workings}
To be done later...

\section{Hooks}
\subsection{ReloadScripts}
\subsection{LuaError}
\subsection{Error}
\subsection{LuaGetLine}
\subsection{Request}

\end{document}
